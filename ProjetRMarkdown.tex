% Options for packages loaded elsewhere
\PassOptionsToPackage{unicode}{hyperref}
\PassOptionsToPackage{hyphens}{url}
%
\documentclass[
]{article}
\usepackage{amsmath,amssymb}
\usepackage{iftex}
\ifPDFTeX
  \usepackage[T1]{fontenc}
  \usepackage[utf8]{inputenc}
  \usepackage{textcomp} % provide euro and other symbols
\else % if luatex or xetex
  \usepackage{unicode-math} % this also loads fontspec
  \defaultfontfeatures{Scale=MatchLowercase}
  \defaultfontfeatures[\rmfamily]{Ligatures=TeX,Scale=1}
\fi
\usepackage{lmodern}
\ifPDFTeX\else
  % xetex/luatex font selection
\fi
% Use upquote if available, for straight quotes in verbatim environments
\IfFileExists{upquote.sty}{\usepackage{upquote}}{}
\IfFileExists{microtype.sty}{% use microtype if available
  \usepackage[]{microtype}
  \UseMicrotypeSet[protrusion]{basicmath} % disable protrusion for tt fonts
}{}
\makeatletter
\@ifundefined{KOMAClassName}{% if non-KOMA class
  \IfFileExists{parskip.sty}{%
    \usepackage{parskip}
  }{% else
    \setlength{\parindent}{0pt}
    \setlength{\parskip}{6pt plus 2pt minus 1pt}}
}{% if KOMA class
  \KOMAoptions{parskip=half}}
\makeatother
\usepackage{xcolor}
\usepackage[margin=1in]{geometry}
\usepackage{color}
\usepackage{fancyvrb}
\newcommand{\VerbBar}{|}
\newcommand{\VERB}{\Verb[commandchars=\\\{\}]}
\DefineVerbatimEnvironment{Highlighting}{Verbatim}{commandchars=\\\{\}}
% Add ',fontsize=\small' for more characters per line
\usepackage{framed}
\definecolor{shadecolor}{RGB}{248,248,248}
\newenvironment{Shaded}{\begin{snugshade}}{\end{snugshade}}
\newcommand{\AlertTok}[1]{\textcolor[rgb]{0.94,0.16,0.16}{#1}}
\newcommand{\AnnotationTok}[1]{\textcolor[rgb]{0.56,0.35,0.01}{\textbf{\textit{#1}}}}
\newcommand{\AttributeTok}[1]{\textcolor[rgb]{0.13,0.29,0.53}{#1}}
\newcommand{\BaseNTok}[1]{\textcolor[rgb]{0.00,0.00,0.81}{#1}}
\newcommand{\BuiltInTok}[1]{#1}
\newcommand{\CharTok}[1]{\textcolor[rgb]{0.31,0.60,0.02}{#1}}
\newcommand{\CommentTok}[1]{\textcolor[rgb]{0.56,0.35,0.01}{\textit{#1}}}
\newcommand{\CommentVarTok}[1]{\textcolor[rgb]{0.56,0.35,0.01}{\textbf{\textit{#1}}}}
\newcommand{\ConstantTok}[1]{\textcolor[rgb]{0.56,0.35,0.01}{#1}}
\newcommand{\ControlFlowTok}[1]{\textcolor[rgb]{0.13,0.29,0.53}{\textbf{#1}}}
\newcommand{\DataTypeTok}[1]{\textcolor[rgb]{0.13,0.29,0.53}{#1}}
\newcommand{\DecValTok}[1]{\textcolor[rgb]{0.00,0.00,0.81}{#1}}
\newcommand{\DocumentationTok}[1]{\textcolor[rgb]{0.56,0.35,0.01}{\textbf{\textit{#1}}}}
\newcommand{\ErrorTok}[1]{\textcolor[rgb]{0.64,0.00,0.00}{\textbf{#1}}}
\newcommand{\ExtensionTok}[1]{#1}
\newcommand{\FloatTok}[1]{\textcolor[rgb]{0.00,0.00,0.81}{#1}}
\newcommand{\FunctionTok}[1]{\textcolor[rgb]{0.13,0.29,0.53}{\textbf{#1}}}
\newcommand{\ImportTok}[1]{#1}
\newcommand{\InformationTok}[1]{\textcolor[rgb]{0.56,0.35,0.01}{\textbf{\textit{#1}}}}
\newcommand{\KeywordTok}[1]{\textcolor[rgb]{0.13,0.29,0.53}{\textbf{#1}}}
\newcommand{\NormalTok}[1]{#1}
\newcommand{\OperatorTok}[1]{\textcolor[rgb]{0.81,0.36,0.00}{\textbf{#1}}}
\newcommand{\OtherTok}[1]{\textcolor[rgb]{0.56,0.35,0.01}{#1}}
\newcommand{\PreprocessorTok}[1]{\textcolor[rgb]{0.56,0.35,0.01}{\textit{#1}}}
\newcommand{\RegionMarkerTok}[1]{#1}
\newcommand{\SpecialCharTok}[1]{\textcolor[rgb]{0.81,0.36,0.00}{\textbf{#1}}}
\newcommand{\SpecialStringTok}[1]{\textcolor[rgb]{0.31,0.60,0.02}{#1}}
\newcommand{\StringTok}[1]{\textcolor[rgb]{0.31,0.60,0.02}{#1}}
\newcommand{\VariableTok}[1]{\textcolor[rgb]{0.00,0.00,0.00}{#1}}
\newcommand{\VerbatimStringTok}[1]{\textcolor[rgb]{0.31,0.60,0.02}{#1}}
\newcommand{\WarningTok}[1]{\textcolor[rgb]{0.56,0.35,0.01}{\textbf{\textit{#1}}}}
\usepackage{graphicx}
\makeatletter
\def\maxwidth{\ifdim\Gin@nat@width>\linewidth\linewidth\else\Gin@nat@width\fi}
\def\maxheight{\ifdim\Gin@nat@height>\textheight\textheight\else\Gin@nat@height\fi}
\makeatother
% Scale images if necessary, so that they will not overflow the page
% margins by default, and it is still possible to overwrite the defaults
% using explicit options in \includegraphics[width, height, ...]{}
\setkeys{Gin}{width=\maxwidth,height=\maxheight,keepaspectratio}
% Set default figure placement to htbp
\makeatletter
\def\fps@figure{htbp}
\makeatother
\setlength{\emergencystretch}{3em} % prevent overfull lines
\providecommand{\tightlist}{%
  \setlength{\itemsep}{0pt}\setlength{\parskip}{0pt}}
\setcounter{secnumdepth}{-\maxdimen} % remove section numbering
\usepackage{sectsty}
\sectionfont{\fontsize{20}{25}\selectfont}
\subsectionfont{\fontsize{16}{20}\selectfont}
\ifLuaTeX
  \usepackage{selnolig}  % disable illegal ligatures
\fi
\IfFileExists{bookmark.sty}{\usepackage{bookmark}}{\usepackage{hyperref}}
\IfFileExists{xurl.sty}{\usepackage{xurl}}{} % add URL line breaks if available
\urlstyle{same}
\hypersetup{
  pdftitle={Projet : Super Market Sales Visualizations},
  pdfauthor={Tek-Up School},
  hidelinks,
  pdfcreator={LaTeX via pandoc}}

\title{Projet : Super Market Sales Visualizations}
\author{Tek-Up School}
\date{2023-2024}

\begin{document}
\maketitle

\vspace{4.5cm}

\hypertarget{pruxe9sentuxe9-par}{%
\subsection{Présenté par :}\label{pruxe9sentuxe9-par}}

\begin{itemize}
\tightlist
\item
  Nour Baklouti
\item
  Rania Dridi
\end{itemize}

\vspace{0.5cm}

\hypertarget{encadruxe9-par}{%
\subsection{Encadré par :}\label{encadruxe9-par}}

\begin{itemize}
\tightlist
\item
  Dr.~Nadjib Mohamed Mehdi BENDAOUD
\end{itemize}

\vfill

\begin{center}\rule{0.5\linewidth}{0.5pt}\end{center}

\hypertarget{supermarket-data-visualization}{%
\subsection{Supermarket Data
Visualization}\label{supermarket-data-visualization}}

Ce rapport vise à visualiser et à analyser :

1.La relation des clients avec le supermarché.

2.Les méthodes de paiement utilisées dans le supermarché.

3.La relation entre les produits et les quantités.

4.Les produits et leurs évaluations.

5.Les types de produits et leurs ventes.

=\textgreater{} Pour plus d'informations, vous trouverez ci-joint le
lien vers notre notebook Kaggle.

\url{https://www.kaggle.com/code/nourbaklouti/bi-supermarket}.

\hypertarget{ruxe9sumuxe9-du-dataframe}{%
\subsection{Résumé du DataFrame}\label{ruxe9sumuxe9-du-dataframe}}

\begin{Shaded}
\begin{Highlighting}[]
\FunctionTok{library}\NormalTok{(ggplot2)}
\end{Highlighting}
\end{Shaded}

\begin{verbatim}
## Warning: le package 'ggplot2' a été compilé avec la version R 4.3.3
\end{verbatim}

\begin{Shaded}
\begin{Highlighting}[]
\FunctionTok{library}\NormalTok{(tidyverse) }
\end{Highlighting}
\end{Shaded}

\begin{verbatim}
## Warning: le package 'tidyverse' a été compilé avec la version R 4.3.3
\end{verbatim}

\begin{verbatim}
## Warning: le package 'tidyr' a été compilé avec la version R 4.3.2
\end{verbatim}

\begin{verbatim}
## Warning: le package 'readr' a été compilé avec la version R 4.3.2
\end{verbatim}

\begin{verbatim}
## Warning: le package 'purrr' a été compilé avec la version R 4.3.2
\end{verbatim}

\begin{verbatim}
## Warning: le package 'dplyr' a été compilé avec la version R 4.3.2
\end{verbatim}

\begin{verbatim}
## Warning: le package 'stringr' a été compilé avec la version R 4.3.2
\end{verbatim}

\begin{verbatim}
## Warning: le package 'forcats' a été compilé avec la version R 4.3.2
\end{verbatim}

\begin{verbatim}
## Warning: le package 'lubridate' a été compilé avec la version R 4.3.3
\end{verbatim}

\begin{verbatim}
## -- Attaching core tidyverse packages ------------------------ tidyverse 2.0.0 --
## v dplyr     1.1.3     v readr     2.1.4
## v forcats   1.0.0     v stringr   1.5.0
## v lubridate 1.9.3     v tibble    3.2.1
## v purrr     1.0.2     v tidyr     1.3.0
## -- Conflicts ------------------------------------------ tidyverse_conflicts() --
## x dplyr::filter() masks stats::filter()
## x dplyr::lag()    masks stats::lag()
## i Use the conflicted package (<http://conflicted.r-lib.org/>) to force all conflicts to become errors
\end{verbatim}

\begin{Shaded}
\begin{Highlighting}[]
\NormalTok{basesm}\OtherTok{\textless{}{-}}\FunctionTok{read.csv}\NormalTok{(}\StringTok{"C:/Users/nourb/Desktop/SupermarketSales/bdsupermarket\_sales.csv"}\NormalTok{)}



\FunctionTok{summary}\NormalTok{(basesm)}
\end{Highlighting}
\end{Shaded}

\begin{verbatim}
##   Invoice.ID           Branch              City           Customer.type     
##  Length:1000        Length:1000        Length:1000        Length:1000       
##  Class :character   Class :character   Class :character   Class :character  
##  Mode  :character   Mode  :character   Mode  :character   Mode  :character  
##                                                                             
##                                                                             
##                                                                             
##     Gender          Product.line         Unit.price       Quantity    
##  Length:1000        Length:1000        Min.   :10.08   Min.   : 1.00  
##  Class :character   Class :character   1st Qu.:32.88   1st Qu.: 3.00  
##  Mode  :character   Mode  :character   Median :55.23   Median : 5.00  
##                                        Mean   :55.67   Mean   : 5.51  
##                                        3rd Qu.:77.94   3rd Qu.: 8.00  
##                                        Max.   :99.96   Max.   :10.00  
##      Tax.5.            Total             Date               Time          
##  Min.   : 0.5085   Min.   :  10.68   Length:1000        Length:1000       
##  1st Qu.: 5.9249   1st Qu.: 124.42   Class :character   Class :character  
##  Median :12.0880   Median : 253.85   Mode  :character   Mode  :character  
##  Mean   :15.3794   Mean   : 322.97                                        
##  3rd Qu.:22.4453   3rd Qu.: 471.35                                        
##  Max.   :49.6500   Max.   :1042.65                                        
##    Payment               cogs        gross.margin.percentage  gross.income    
##  Length:1000        Min.   : 10.17   Min.   :4.762           Min.   : 0.5085  
##  Class :character   1st Qu.:118.50   1st Qu.:4.762           1st Qu.: 5.9249  
##  Mode  :character   Median :241.76   Median :4.762           Median :12.0880  
##                     Mean   :307.59   Mean   :4.762           Mean   :15.3794  
##                     3rd Qu.:448.90   3rd Qu.:4.762           3rd Qu.:22.4453  
##                     Max.   :993.00   Max.   :4.762           Max.   :49.6500  
##      Rating      
##  Min.   : 4.000  
##  1st Qu.: 5.500  
##  Median : 7.000  
##  Mean   : 6.973  
##  3rd Qu.: 8.500  
##  Max.   :10.000
\end{verbatim}

\hypertarget{description-des-variables}{%
\section{Description des variables}\label{description-des-variables}}

Les variables présentes dans le jeu de données sont les suivantes :

\begin{itemize}
\item
  Invoice ID : Identifiant unique de la facture.
\item
  Branch : Succursale du supermarché (A, B, C).
\item
  City :Ville où se trouve le supermarché.
\item
  Customer type :Type de client (Member pour membre, Normal pour
  non-membre).
\item
  Gender : Genre du client (Male pour homme, Female pour femme).
\item
  Product line :Catégorie du produit.
\item
  Unit price : Prix unitaire du produit.
\item
  Quantity :Quantité d'articles achetés.
\item
  Tax 5\% : Montant de la taxe (5\%) sur l'achat.
\item
  Total :Montant total de l'achat (y compris la taxe).
\item
  Date : Date de l'achat .
\item
  Time : Heure de l'achat .
\item
  Payment : Mode de paiement (Ewallet, Cash, Credit card).
\item
  cogs (Cost of Goods Sold) :Coût des biens vendus.
\item
  gross margin percentage :Pourcentage de\\
  marge brute.
\item
  gross income : Revenu brut
\item
  Rating : Évaluation du client pour l'expérience d'achat
\end{itemize}

\hypertarget{visualization-des-donnuxe9es}{%
\section{Visualization des données}\label{visualization-des-donnuxe9es}}

\hypertarget{la-relation-des-clients-avec-le-supermarchuxe9}{%
\subsection{La relation des clients avec le
supermarché:}\label{la-relation-des-clients-avec-le-supermarchuxe9}}

\hypertarget{la-distribution-par-genre}{%
\section{la distribution par genre}\label{la-distribution-par-genre}}

\begin{Shaded}
\begin{Highlighting}[]
\FunctionTok{barplot}\NormalTok{(}\FunctionTok{table}\NormalTok{(basesm}\SpecialCharTok{$}\NormalTok{Gender), }\AttributeTok{col =} \FunctionTok{c}\NormalTok{(}\StringTok{"blue"}\NormalTok{, }\StringTok{"pink"}\NormalTok{),}
        \AttributeTok{main =} \StringTok{"Répartition du genre des clients"}\NormalTok{,}
        \AttributeTok{xlab =} \StringTok{"Genre"}\NormalTok{, }\AttributeTok{ylab =} \StringTok{"Fréquence"}\NormalTok{)}
\end{Highlighting}
\end{Shaded}

\includegraphics{ProjetRMarkdown_files/figure-latex/unnamed-chunk-1-1.pdf}
=\textgreater{} Ce graphique présente la répartition du genre des
clients dans le supermarché, avec les couleurs bleue pour les clients
masculins et rose pour les clients féminins.On peut constater que la
difference entre le nombre des deux sexes des clients n'est pas vraiment
significative.

\hypertarget{la-distribution-du-type-de-client}{%
\section{La distribution du type de
client}\label{la-distribution-du-type-de-client}}

\begin{Shaded}
\begin{Highlighting}[]
\FunctionTok{barplot}\NormalTok{(}\FunctionTok{table}\NormalTok{(basesm}\SpecialCharTok{$}\NormalTok{Customer.type), }\AttributeTok{col =} \FunctionTok{c}\NormalTok{(}\StringTok{"orange"}\NormalTok{, }\StringTok{"yellow"}\NormalTok{, }\StringTok{"green"}\NormalTok{),}
        \AttributeTok{main =} \StringTok{"Répartition des types de clients"}\NormalTok{,}
        \AttributeTok{xlab =} \StringTok{"Type de client"}\NormalTok{, }\AttributeTok{ylab =} \StringTok{"Fréquence"}\NormalTok{)}
\end{Highlighting}
\end{Shaded}

\includegraphics{ProjetRMarkdown_files/figure-latex/unnamed-chunk-2-1.pdf}

=\textgreater{} Ce graphique illustre la répartition des types de
clients dans le supermarché, distinguant entre les membres (Member) et
les clients non-membres (Normal). On conclut qu'il ya autant des membres
que des clients normaux.

\hypertarget{la-distribution-du-mode-de-paiement-par-genre}{%
\section{La distribution du mode de paiement par
genre}\label{la-distribution-du-mode-de-paiement-par-genre}}

\begin{Shaded}
\begin{Highlighting}[]
\FunctionTok{ggplot}\NormalTok{(basesm, }\FunctionTok{aes}\NormalTok{(}\AttributeTok{x =}\NormalTok{ Gender, }\AttributeTok{fill =}\NormalTok{ Payment)) }\SpecialCharTok{+}
  \FunctionTok{geom\_bar}\NormalTok{(}\AttributeTok{position =} \StringTok{"stack"}\NormalTok{) }\SpecialCharTok{+}
  \FunctionTok{labs}\NormalTok{(}\AttributeTok{title =} \StringTok{"Répartition du mode de paiement par genre"}\NormalTok{,}
       \AttributeTok{x =} \StringTok{"Genre"}\NormalTok{,}
       \AttributeTok{y =} \StringTok{"Nombre de transactions"}\NormalTok{,}
       \AttributeTok{fill =} \StringTok{"Mode de paiement"}\NormalTok{)}
\end{Highlighting}
\end{Shaded}

\includegraphics{ProjetRMarkdown_files/figure-latex/unnamed-chunk-3-1.pdf}
=\textgreater{} Ce graphique présente la répartition du mode de paiement
utilisé par genre de client, empilant les différents modes de paiement
pour chaque genre. On peut visualiser que les hommes paient un peu plus
que les femmes avec leurs Ewallet pendant que les femmes utilisent plus
du cash. \#\# Les méthodes de paiement utilsées dans le supermarché: \#
La distribution des modes de paiement.

\begin{Shaded}
\begin{Highlighting}[]
\FunctionTok{barplot}\NormalTok{(}\FunctionTok{table}\NormalTok{(basesm}\SpecialCharTok{$}\NormalTok{Payment), }\AttributeTok{col =} \FunctionTok{c}\NormalTok{(}\StringTok{"blue"}\NormalTok{, }\StringTok{"green"}\NormalTok{, }\StringTok{"red"}\NormalTok{, }\StringTok{"yellow"}\NormalTok{),}
        \AttributeTok{main =} \StringTok{"Répartition des types de paiements"}\NormalTok{,}
        \AttributeTok{xlab =} \StringTok{"Mode de paiement"}\NormalTok{, }\AttributeTok{ylab =} \StringTok{"Fréquence"}\NormalTok{)}
\end{Highlighting}
\end{Shaded}

\includegraphics{ProjetRMarkdown_files/figure-latex/unnamed-chunk-4-1.pdf}

\begin{Shaded}
\begin{Highlighting}[]
\NormalTok{      fill }\OtherTok{=} \StringTok{"Mode de paiement"}
\end{Highlighting}
\end{Shaded}

=\textgreater{} Ce graphique montre la répartition des différents modes
de paiement utilisés dans le supermarché. L'utilisation de la Ewallet et
Cash sont les plus courrants et la credit card et un peu moins utilisées
pour le paiement.

\hypertarget{lhistogramme-des-prix-unitaires}{%
\section{L'histogramme des prix
unitaires}\label{lhistogramme-des-prix-unitaires}}

\begin{Shaded}
\begin{Highlighting}[]
\FunctionTok{hist}\NormalTok{(basesm}\SpecialCharTok{$}\NormalTok{Unit.price, }\AttributeTok{col =} \StringTok{"skyblue"}\NormalTok{,}
     \AttributeTok{main =} \StringTok{"Répartition des prix unitaires"}\NormalTok{,}
     \AttributeTok{xlab =} \StringTok{"Prix unitaire"}\NormalTok{,}
     \AttributeTok{ylab =} \StringTok{"Fréquence"}\NormalTok{)}
\end{Highlighting}
\end{Shaded}

\includegraphics{ProjetRMarkdown_files/figure-latex/unnamed-chunk-5-1.pdf}
=\textgreater{} Cet histogramme présente la distribution des prix
unitaires des produits vendus dans le supermarché.Les Poduits les plus
vendus sont les produit les plus chères , sinon tout les produits sont
vendus d'une frequence entre 100 et 120.

\hypertarget{plot-the-distribution-of-payment-modes-by-customer-type}{%
\section{Plot the distribution of Payment modes by Customer
type}\label{plot-the-distribution-of-payment-modes-by-customer-type}}

\begin{Shaded}
\begin{Highlighting}[]
\FunctionTok{ggplot}\NormalTok{(basesm, }\FunctionTok{aes}\NormalTok{(}\AttributeTok{x =} \StringTok{""}\NormalTok{, }\AttributeTok{fill =} \StringTok{\textasciigrave{}}\AttributeTok{Customer.type}\StringTok{\textasciigrave{}}\NormalTok{)) }\SpecialCharTok{+}
  \FunctionTok{geom\_bar}\NormalTok{(}\AttributeTok{width =} \DecValTok{1}\NormalTok{) }\SpecialCharTok{+}
  \FunctionTok{coord\_polar}\NormalTok{(}\AttributeTok{theta =} \StringTok{"y"}\NormalTok{) }\SpecialCharTok{+}
  \FunctionTok{labs}\NormalTok{(}\AttributeTok{title =} \StringTok{"Répartition du mode de paiement par type de client"}\NormalTok{,}
       \AttributeTok{fill =} \StringTok{"Type de client"}\NormalTok{,}
       \AttributeTok{x =} \ConstantTok{NULL}\NormalTok{,}
       \AttributeTok{y =} \ConstantTok{NULL}\NormalTok{)}
\end{Highlighting}
\end{Shaded}

\includegraphics{ProjetRMarkdown_files/figure-latex/unnamed-chunk-6-1.pdf}

=\textgreater{} Ce graphique montre la répartition du mode de paiement
par type de client, en affichant les modes de paiement sous forme de
diagramme en secteurs, séparés par type de client. N'importe qu'il soit
le type du clients , les méthodes de paiements restent identiques.

\hypertarget{la-relation-entre-les-produits-et-les-quantituxe9s}{%
\subsection{La relation entre les produits et les
quantités:}\label{la-relation-entre-les-produits-et-les-quantituxe9s}}

\hypertarget{plot-the-boxplot-of-total-purchase-amount-by-product-line}{%
\section{Plot the boxplot of Total purchase amount by Product
line}\label{plot-the-boxplot-of-total-purchase-amount-by-product-line}}

\begin{Shaded}
\begin{Highlighting}[]
\FunctionTok{ggplot}\NormalTok{(basesm, }\FunctionTok{aes}\NormalTok{(}\AttributeTok{x =}\NormalTok{ Product.line, }\AttributeTok{y =}\NormalTok{ Total, }\AttributeTok{fill =}\NormalTok{ Product.line)) }\SpecialCharTok{+}
  \FunctionTok{geom\_boxplot}\NormalTok{() }\SpecialCharTok{+}
  \FunctionTok{labs}\NormalTok{(}\AttributeTok{title =} \StringTok{"Montant total de l\textquotesingle{}achat par catégorie de produit"}\NormalTok{,}
       \AttributeTok{x =} \StringTok{"Catégorie de produit"}\NormalTok{,}
       \AttributeTok{y =} \StringTok{"Montant total de l\textquotesingle{}achat"}\NormalTok{)}
\end{Highlighting}
\end{Shaded}

\includegraphics{ProjetRMarkdown_files/figure-latex/unnamed-chunk-7-1.pdf}
=\textgreater{} Ce graphique présente la répartition du montant total de
l'achat par catégorie de produit sous forme de boîtes à moustaches,
permettant de visualiser les écarts entre les différentes catégories.
Toutes les categories sont ont des montant d'achat proches , mais Home
and LifeStyle et Health and Beauty sont les catégories les plus vendus
suivi par Electronic accessories.

\hypertarget{plot-the-histogram-of-quantity}{%
\section{Plot the histogram of
Quantity}\label{plot-the-histogram-of-quantity}}

\begin{Shaded}
\begin{Highlighting}[]
\FunctionTok{ggplot}\NormalTok{(basesm, }\FunctionTok{aes}\NormalTok{(}\AttributeTok{x =}\NormalTok{ Quantity)) }\SpecialCharTok{+}
  \FunctionTok{geom\_histogram}\NormalTok{(}\AttributeTok{fill =} \StringTok{"skyblue"}\NormalTok{, }\AttributeTok{color =} \StringTok{"black"}\NormalTok{, }\AttributeTok{bins =} \DecValTok{30}\NormalTok{) }\SpecialCharTok{+}
  \FunctionTok{labs}\NormalTok{(}\AttributeTok{title =} \StringTok{"Distribution de la quantité d\textquotesingle{}articles achetés"}\NormalTok{,}
       \AttributeTok{x =} \StringTok{"Quantité d\textquotesingle{}articles"}\NormalTok{,}
       \AttributeTok{y =} \StringTok{"Fréquence"}\NormalTok{)}
\end{Highlighting}
\end{Shaded}

\includegraphics{ProjetRMarkdown_files/figure-latex/unnamed-chunk-8-1.pdf}

=\textgreater{} Cet histogramme illustre la distribution de la quantité
d'articles achetés dans le supermarché. La fréquence des quantité
d'articles égale à 10 est la plus supérieur pendant que la la quantité
d'article égale à 8 a la plus faible frequence.

\hypertarget{plot-the-histogram-of-gross-income}{%
\section{Plot the histogram of Gross
income}\label{plot-the-histogram-of-gross-income}}

\begin{Shaded}
\begin{Highlighting}[]
\FunctionTok{ggplot}\NormalTok{(basesm, }\FunctionTok{aes}\NormalTok{(}\AttributeTok{x =}\NormalTok{ gross.income)) }\SpecialCharTok{+}
  \FunctionTok{geom\_histogram}\NormalTok{(}\AttributeTok{fill =} \StringTok{"lightgreen"}\NormalTok{, }\AttributeTok{color =} \StringTok{"black"}\NormalTok{, }\AttributeTok{bins =} \DecValTok{30}\NormalTok{) }\SpecialCharTok{+}
  \FunctionTok{labs}\NormalTok{(}\AttributeTok{title =} \StringTok{"Distribution des revenus bruts"}\NormalTok{,}
       \AttributeTok{x =} \StringTok{"Revenu brut"}\NormalTok{,}
       \AttributeTok{y =} \StringTok{"Fréquence"}\NormalTok{)}
\end{Highlighting}
\end{Shaded}

\includegraphics{ProjetRMarkdown_files/figure-latex/unnamed-chunk-9-1.pdf}

=\textgreater{} Cet histogramme présente la distribution des revenus
bruts générés par les ventes de produits dans le supermarché. Plus le
revenue brute croit plus la frequence decroit .

\hypertarget{les-produits-et-leurs-uxe9valuations}{%
\subsection{Les produits et leurs
évaluations:}\label{les-produits-et-leurs-uxe9valuations}}

\hypertarget{plot-the-histogram-of-rating}{%
\section{Plot the histogram of
Rating}\label{plot-the-histogram-of-rating}}

\begin{Shaded}
\begin{Highlighting}[]
\FunctionTok{ggplot}\NormalTok{(basesm, }\FunctionTok{aes}\NormalTok{(}\AttributeTok{x =}\NormalTok{ Rating)) }\SpecialCharTok{+}
  \FunctionTok{geom\_histogram}\NormalTok{(}\AttributeTok{fill =} \StringTok{"orange"}\NormalTok{, }\AttributeTok{color =} \StringTok{"black"}\NormalTok{, }\AttributeTok{bins =} \DecValTok{30}\NormalTok{) }\SpecialCharTok{+}
  \FunctionTok{labs}\NormalTok{(}\AttributeTok{title =} \StringTok{"Distribution des évaluations des clients"}\NormalTok{,}
       \AttributeTok{x =} \StringTok{"Évaluation"}\NormalTok{,}
       \AttributeTok{y =} \StringTok{"Fréquence"}\NormalTok{)}
\end{Highlighting}
\end{Shaded}

\includegraphics{ProjetRMarkdown_files/figure-latex/unnamed-chunk-10-1.pdf}

=\textgreater{} Cet histogramme illustre la distribution des évaluations
données par les clients pour leur expérience d'achat dans le
supermarché. la note la plus donnée est égale a 6 . c'est un peut rare
pour un client qu'il donne la note 4 ou 10.

\hypertarget{la-dispersion-entre-le-mode-de-paiement-et-la-notation.}{%
\section{La dispersion entre le mode de paiement et la
notation.}\label{la-dispersion-entre-le-mode-de-paiement-et-la-notation.}}

\begin{Shaded}
\begin{Highlighting}[]
\FunctionTok{ggplot}\NormalTok{(basesm) }\SpecialCharTok{+}
  \FunctionTok{geom\_point}\NormalTok{(}\FunctionTok{aes}\NormalTok{(}\AttributeTok{x =}\NormalTok{ Payment, }\AttributeTok{y =}\NormalTok{ Rating)) }\SpecialCharTok{+}
  \FunctionTok{labs}\NormalTok{(}\AttributeTok{title =} \StringTok{"Nuage de points entre le mode de paiement et l\textquotesingle{}évaluation"}\NormalTok{,}
       \AttributeTok{x =} \StringTok{"Mode de paiement"}\NormalTok{,}
       \AttributeTok{y =} \StringTok{"Évaluation"}\NormalTok{)}
\end{Highlighting}
\end{Shaded}

\includegraphics{ProjetRMarkdown_files/figure-latex/unnamed-chunk-11-1.pdf}

=\textgreater{} Ce nuage de points montre la relation entre le mode de
paiement utilisé et l'évaluation donnée par les clients pour leur
expérience d'achat. Les clients qui paient avec la credit card ou
Ewallet donnent les meilleurs evaluations.

\hypertarget{les-types-de-produits-et-leurs-ventes}{%
\subsection{Les types de produits et leurs
ventes:}\label{les-types-de-produits-et-leurs-ventes}}

\hypertarget{la-distribution-des-lignes-de-produits}{%
\section{La distribution des lignes de
produits}\label{la-distribution-des-lignes-de-produits}}

\begin{Shaded}
\begin{Highlighting}[]
\FunctionTok{barplot}\NormalTok{(}\FunctionTok{table}\NormalTok{(basesm}\SpecialCharTok{$}\NormalTok{Product.line))}
\end{Highlighting}
\end{Shaded}

\includegraphics{ProjetRMarkdown_files/figure-latex/unnamed-chunk-12-1.pdf}

=\textgreater{} Ce barplot présente la répartition des différents types
de produits vendus dans le supermarché.L'existance des produits du
Fashion Accesoiries est la plus remarquable pendant qu'il n ya pas trop
de produit de Health and Beauty.

\hypertarget{la-distribution-des-transactions-par-branche.}{%
\section{la distribution des transactions par
branche.}\label{la-distribution-des-transactions-par-branche.}}

\begin{Shaded}
\begin{Highlighting}[]
\FunctionTok{ggplot}\NormalTok{(basesm, }\FunctionTok{aes}\NormalTok{(}\AttributeTok{x =}\NormalTok{ Branch)) }\SpecialCharTok{+}
  \FunctionTok{geom\_bar}\NormalTok{(}\AttributeTok{fill =} \StringTok{"lightblue"}\NormalTok{) }\SpecialCharTok{+}
  \FunctionTok{labs}\NormalTok{(}\AttributeTok{title =} \StringTok{"Nombre de transactions par succursale"}\NormalTok{,}
       \AttributeTok{x =} \StringTok{"Succursale"}\NormalTok{,}
       \AttributeTok{y =} \StringTok{"Nombre de transactions"}\NormalTok{)}
\end{Highlighting}
\end{Shaded}

\includegraphics{ProjetRMarkdown_files/figure-latex/unnamed-chunk-13-1.pdf}

=\textgreater{} Ce graphique montre le nombre de transactions effectuées
dans chaque succursale du supermarché. La succursale A a le plus nombre
de transaction pendant que C a le nombre de transaction le plus
inferieur.

\hypertarget{conclusion}{%
\section{Conclusion}\label{conclusion}}

En conclusion, l'analyse des ventes pour ce supermarché révèle une
croissance constante dans les grandes villes, avec une concurrence
accrue sur le marché. Les données historiques sur les ventes dans les
trois succursales pendant trois mois fournissent une base solide pour
l'application de méthodes d'analyse prédictive, facilitant ainsi la
prise de décisions éclairées pour optimiser les performances et rester
compétitif sur le marché des supermarchés.

\end{document}
